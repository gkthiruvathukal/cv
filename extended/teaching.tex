\newpage
\section{Teaching Summary}

I am a passionate teacher focused on blending the tradition of computing with modern practice. At Loyola University Chicago (where I have been continuously since 2001) I have focused my innovation efforts primarily on courses that integrate foundational principles with advanced research topics. Being trained in computer sciences and sciences as an undergraduate, I have always thought about computing in interdisciplinary terms and am not shy about learning and teaching new topics as my career has progressed.
\vspace{5pt}

Since the mid-2000s, I have also rediscovering a love of introductory teaching, with a strong focus on undergraduate teaching at the 100 and 200 level. This culminated in receiving Loyola's most prestigious teaching honor, the Sujack teaching excellence award, in 2009 (it also carries a cash prize, endowed by the Sujack family). I'm also particularly keen to teach courses that impart general knowledge. I've been active in teaching first year courses to majors and non-majors alike and advanced courses, primarily in software engineering and systems areas (details below). 
\vspace{5pt}

I've also been instrumental in developing interdisciplinary and core knowledge courses, including gaining approval for such courses at college and university level councils, including in historical knowledge and arts/humanities areas. I am known to be \emph{the} faculty member who has collaborated across multiple disciplines, including in the sciences and humanities. I co-founded the Center for Textual Studies and Digital Humanities at Loyola University Chicago and created one of Chicago's first--if not the first--MA programs in Digital Humanities and the MS IT program in collaboration with the Quinlan Business school. I was also involved in early discussions about the BS programs in Bioinformatics and Data Science.
\vspace{5pt}

In sum, despite my strong focus on research, I have devoted significant time to embracing the institutional reality in my present university by being a curricular innovator and integrating research and teaching at every step of the way. Given a significant number of papers co-authored by students who were undergraduates or graduates in my classes, my curricular efforts have been greatly helpful to maintain a productive research career in an non-doctoral non-R1 university department.
\vspace{5pt}

\subsection{Courses Developed, Refactored, and Taught}


\cvline{Distributed Systems (COMP 339-439)}{Distributed systems have long been taught (and even written about) in the most abstract conceptual and theoretical terms. In this course, I focus on the tradition of distributed systems but supplement it with an exploration of actual distributed systems. Because the area is so vast, this course allows each group to choose from a set of projects or propose one of their own. Using the an agile software developemnt paradigm, an integrated approach is taken to devising requirements, testing, and development that allows projects to be completed in a timely manner. As projects are proposed and developed, care is taken every step of the way to assess how the projects measure against the established concepts and theory in distributed systems. Technologies (e.g. DNS, LDAP, databases, and other server software) are evaluated as well to understand the trade-offs that must be made in order to realize a working distributed framework/application in practice. I'm in the process of preparing for the next offering, which will include a detailed examination of modern systems, including torrents, blockchains, and clouds/containers.}

\cvline{High Performance Computing (Loyola COMP 364/464)}{A peculiar irony is that my doctoral dissertation is actually in the area of high-performance computing. Yet, because our department has  3 other faculty who have worked in this area, I have only taught this course once, so I had to make a real impact. So it was exciting when I got the opportunity to teach this course in Fall 2009, and the student response/evaluations was overwhelmingly positive. In this course, students learned both the tradition and modern practice of high-performance computing, which is a topic that played a major role in the modern phenomenon known as cloud computing. Everyone knows about ``the cloud'' via applications coming from Google, Yahoo!, Microsoft, Facebook, and others. Since this offering, I have been teaching other important courses, but I continue to advise our part-time instructor on ways to keep this course exciting, including working with GPGPUs and national supercomputing resources, XSEDE, and those at other national laboratories, such as Argonne, where I hold a visiting faculty appointment.}
\vspace{5pt}

\cvline{Software Engineering (COMP 330)}{In recent years, I have been in charge of the undergraduate course in Software Engineering. This has come about as a result of my latest research in software engineering, which focuses on bringing software engineering methods to support the development of advanced research software. In this course, I have also pioneered methods for how to do teamwork from the beginning. We start by organizing the class in teams of two, which builds on earlier curricular ideas of pair programming but with higher expectations. Contrasted with many SE courses of the past, the course (and I) takes a more holistic approach for how to evaluate projects. Working software is important, but including ideas of software engineering is what earns the most points. Building on Gardner's Theory of Multiple Intelligences, the goal is to bring out individual strengths, which can be chosen from a set of predtermined (but not limiting) categories. Elements of requirements, design, testing, version control, visual design, etc., are rated. There is theoretically no limit to the number of points that a team can earn, but teams are also compared for their ability to deliver in categories besides coding. The end result is a software engineering class where the concepts are actually applied and observed in student project work.}


\cvline{Free and Open Source Practicum (COMP 312-412)}{Linux is based on an old idea: Unix. Nevertheless, never before has an operating system triggered as much community interest and innovation as the Linux platform. This course provides an in-depth exploration of the Linux platform  and open source software. Topics to be addressed at a minimum include do-it-yourself (DIY) personal computing, system installation, configuration, performance tuning, scripting languages, server software, file systems, authentication schemes, and kernel hacking (among others). Small teams will be assembled to do a number of advanced projects in Linux. Similar to the Distributed Systems course, Linux is a complete platform for conducting a myriad of systems experiments, so the best way to tackle it is via focused student projects, where students will have the opportunity to learn many aspects of Linux but accomplish something of significance as a final outcome. Recent offerings have also looked more closely at licensing issues. For programming, we strongly emphasize the use of Python, which is a great language for bringing more diverse individuals to computing. It also allows us to structure this course with minimal prerequisites.}


\cvline{Markup Languages and Applications (Loyola COMP 336-436)}{Extensible Markup Language is a complex framework for self-describing documents, data interchange, and advanced data processing through transformation. In this course, we explore the XML framework and design patterns/techniques for making the most effective use of it. The work breakdown is a mix of traditional small to medium size programming assignments that progresses toward independent projects and opportunity to explore specific aspects of XML in greater depth. Programming is done in Python (an excellent language for text processing with very high level data structures), Java, and XML itself (e.g. through XSLT). Students learn all aspects of working with XML: modeling with schemas, parsing, data binding, generating XML, and transformation.}


\cvline{History of Computing (Loyola COMP 111)}{Proposed, developed, and taught a core historical knowledge course about the history of early mathematical, calculating, and computing history. This course exposes students to history through the lens of technology, focused primarily on the period from 1670-Present. The course also introduces students to mathematical and computing concepts that predate Western Civilization as we know it, including but not limited to the Babylonians, Indus Valley, Inca, etc., who are largely credited with introducing most ideas of arithmetic to the West. Since its inaugural offering in 2005, the course has introduced hundreds of students to computing ideas without requiring them to become computer scientists. In addition, computer science students have taken this course, which allows them to meet one of their core History requirements and become more knowledgeable about their profession. (Many CS students, especially today, do not realize the incredible history of our discipline, let alone knowing the key individuals and historical events, e.g. WW II and cracking the Enigma, where computers and computational thinkers played a major role.}

\vspace{5pt}
\cvline{Science and Society (Loyola Honors Program 204)}{I occasionally teach the Science and Society course in the Interdisciplinary Honors Program. The theme for this course is \emph{networks}, which includes but is not limited to topic of social networking. This course is being based around an exciting textbook by Earley and Kleinberg, Networks, Crowds, and Markets, that is the basis for a quad-disciplinary class at Cornell (in CS, Economics, Sociology, and Political Science).}

\vspace{5pt}


\subsection{Other Regular Department Courses Taught}

In addition to developing the above courses, I have also participated extensively in teaching core computer science courses within the undergraduate and graduate curriculum:

\cvlistitem{COMP 125 (3 credits): Visual Information Processing (Core Course)}
\cvlistitem{COMP 150 (3 credits): Introduction to Computer Science (Principles-Focused Majors Course)}
\cvlistitem{COMP 170 (3 credits): Introduction to Computer Science (Programming-Focused Majors Course)}
\cvlistitem{COMP 272 (3 credits): Data Structures and Object Oriented Programming using C++}
\cvlistitem{COMP 343/443 (3 credits): Introduction to Computer Networks}
\cvlistitem{COMP 372 (3 credits): Programming Languages}
\cvlistitem{COMP 310/410 (3 credits): Advanced Operating Systems}

In the case of COMP 170, I co-developed an online textbook as part of our 3-year experiment to evaluate C\# as the teaching language. This course is available at \href{http://books.cs.luc.edu/introcs-csharp/}{books.cs.luc.edu/introcs-csharp/}. Although the department decided to return to Java, the book remains a popular resource for students who want to learn C\#.

\subsection{Evidence of Teaching Effectiveness}

I have a strong track record of teaching in my entire academic career. Since joining Loyola, I have consistently received excellent student evaluations mostly within the top 10-15\% of the department. I don't personally place much weight on teaching evaluations but am grateful that my students appreciate my teaching, even when their final grades isn't always what they would like. Many students take more than one class with me--often within the same semester or academic term. This is particularly true in my systems classes. As an example, in a recent offering of Operating Systems and Distributed Systems, nearly 1/2 of the class are students coming from Operating Systems. Distributed Systems is not a required class, so this speaks to their enthusiasm for me as an instructor. 
\vspace{5pt}

Beyond didactic class offerings, I am among the most contracted professors for independent study since oining the department, including undergraduate and graduate students alike. Many of the students--undergraduate and graduate--use independent study opportunities to explore advanced graduate masters and doctoral level work. Those who do not use it to pursue employment in more research-minded companies within and outside of the Chicago area. 
\vspace{5pt}

I am routinely approached by employers who want to hire students who have been through my classes in particular, and students in the department know that I have a good pulse on opportunities for meaningful and long-term employment
\vspace{5pt}

In AY 2008-2009, I was a first-time nominee and winner of the competitive Sujack Teaching Excellence Award, which is given in the College of Arts and Sciences but is also widely regarded as the highest teaching award at Loyola University Chicago. This award is not based on teaching evaluations. Students nominate professors. A committee then requests materials from nominees. Although teaching data are submitted as partial evidence, the award is usually won or lost based on curricular materials developed by the faculty member. Great consideration is placed on pedagogical approach and student outcomes. Beyond receiving the recognition at a special ceremony, there is also a cash prize. Although I wasn't in it for the money, there is something about receiving actual cash that establishes a permanent cache for this award.
\vspace{5pt}


\cvline{Undergraduate/Graduate Advising}{ I have a long-term track record of advising undergraduate and graduate students in computer science and beyond, including a significant number of publications where students were actively mentored and participated in as co-authors in published research. In addition, I am known since coming to Loyola in 2001 as an approachable professor who spends a significant number hours assisting students in their advising needs and providing general overall direction, which contributed in part to my selection as one of the two Sujack teaching excellence award winners for AY 2008-2009. }
\vspace{5pt}


\cvline{Professional and Student Societies}{ I have served as the co-advisor of our Association for Computing Machinery chapter from 2001-2004. The ACM is the world's largest and most prestigious computing organization, focused on the academic and professional practice of computing. In recent years, I have served as the faculty advisor for Free and Open Source Software at Loyola (FOSSAL). }
\vspace{5pt}


\cvline{Professional and Industry Liaison}{ Among CS faculty, I am among the most experienced in the computing profession, having served as a development manager in a Fortune 250 company (R.R. Don nelley and Sons) and having been involved in a couple of startups, one of which was funded via the DARPA SBIR program. For this reason, students actively seek my input on understanding the profession and making career-defining decisions, such as finding an appropriate job. In addition, I have been instrumental in providing direct referrals of our graduates (undergraduate and graduate) to reputable companies in the Chicago area and around the country. }
\vspace{5pt}


\cvline{Publications with Student Co-Authors}{ I have a long-term record of involving Loyola University students in my research projects. Many have been funded on grants, while many have not. This is a selection of publications since 2005 involving Loyola University Chicago student co-authors (shown in italics). The entire list of publications also highlights LUC student co-authors in italics. }
