\subsection{Scholarly Notetaking Systems and A Cultural History of Computing Book (Coming 2019)}

This is a collaboration with David B. Dennis, Professor of History at Loyola University Chicago.
\vspace{5pt}

Before the computer revolution, great works of scholarship in the social sciences and humanities were written using paper-based systems of notes, cardfiles, and outlines.  But no electronic tools have fully emulated classic principles of research and writing. We have sought to synthesize the best practices of our predecessors with the best practices of computerization in a new program, ZettelGeist, that supports Zettelkasten, a general method for organizing, indexing, and updating notes.
\vspace{5pt}

Co-created by a myself (a computer scientist) and David Dennis (a cultural historian), our Zettelkasten tool (hereafter ZettelGeist) is an open source, plaintext notetaking system inspired by index cards. In ZettelGeist, Zetteln (the German word for note cards, or zettels in English) are written in YAML or Markdown--slightly modified to include Zettel metadata fields--format, a simplified markup system that allows for easy entry of structured data using any text editor. Similar to traditional index cards, zettels can be organized to include as much or as little information as desired, allowing for more organic and granular generation of a text. Zettels can be transformed and searched with the ZettelGeist tools to support the research process. Commercial “notetaking systems” today are hobbled by the problem of having to copy/paste from research files to draft documents, thereby disconnecting notes as the source of truth. In ZettelGeist, all writing can be done using zettels themselves, which can then be included in a book (or any document)--verbatim or transformed via queries written in a predicate calculus language, similar to Google Mail (Gmail) search operators. ZettelGeist is not (yet) a WYSIWYG system. It is distributed as an API (Application Programming Interface) and a set of straightforward command-line tools in keeping with the Unix philosophy.
\vspace{5pt}

The core abstractions of ZettelGeist allow us to maintain manuscripts using vim, LaTeX, git (for distributed version control and collaborative note-taking), Zotero, and ZettelGeist. ZettelGeist is 100\% free software and can be installed via Python’s PyPI system. Speaking specifically to git, because zettels are represented in plaintext using established open text formats with robust tools (e.g. pandoc), the entire workflow of preparing a manuscript can be managed using git hosting solutions (e.g. GitHub), which we use for all aspects of scholarship. 
\vspace{5pt}

The actual use case for the tools is our forthcoming book, \emph{The History of Computing and Its Cultures}, which will be published by Taylor and Francis in 2019. While a number of textbooks and surveys have traced stages of computing history from the ancients to the web in traditional narrative fashion, none has broadly applied critical historiographical methods to explore the relationships between these developments and their social and cultural contexts. Ours provides a much-needed synthesis, explaining that the history of technology—and the technology of information in particular—is intricately woven together with the cultural history of humanity, as well as its social and political transformations.
\vspace{5pt}

Taking this omni-disciplinary approach, our work delineates the fundamentals everyone needs to know about computing history from the evolution of number systems, through the invention of calculating and computing machines, to the emergence of communication technology via the Internet. More than any other survey of computing history, though, this book contextualizes each major technological advancements with correlate developments in politics, economics, literature, art, music, film, and many more areas. 
\vspace{5pt}

In essence, the book blends a Western Civilization-Humanities Survey with a History of  Computing that will help readers, especially students, place themselves conceptually into the context of the ongoing information revolution that is human cultural history. A brief initial survey of early computing from the ancients to the enlightenment is intended to provide foundations for more in-depth coverage of innovations in the 19th century, when the modern information age really began. Commencing its main coverage at this stage, this book then highlights processes that led to the most recent developments of the last half century.

\begin{refsection}
    \nocite{dennis_thiruvathukal_dh2019,david_b._dennis_computer_2017}
    \printbibliography[heading=none]
\end{refsection}

