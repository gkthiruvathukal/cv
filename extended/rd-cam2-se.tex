\subsection{CAM$^2$: Continuous Analysis of Many CAMeras}

This is a collaboration with Yung-Hsiang Lu, Professor of Computer Engineering at Purdue University.
\vspace{5pt}

Many network cameras have been deployed for a wide range of purposes, such as monitoring traffic, evaluating air pollution, observing wildlife, and watching landmarks. The data from these cameras can provide rich information about the natural environment and human activities. To extract valuable information from this network of cameras, complex computer programs are needed to retrieve data from the geographically distributed cameras and to analyze the data. This project creates a open source software infrastructure by solving many problems common to different types of analysis programs. By using this infrastructure, researchers can focus on scientific discovery, not writing computer programs. This project can improve efficiency and thus reduce the cost for running programs analyzing large amounts of data. This infrastructure promotes education because students can obtain an instantaneous view of the network cameras and use the visual information to understand the world. Better understanding of the world may encourage innovative solutions for many pressing issues, such as better urban planning and lower air pollution. This project can enhance diversity through multiple established programs that encourage underrepresented minorities to pursue careers in science and engineering.
\vspace{5pt}

This project will combine: (1) the ability to retrieve data from many heterogeneous and distributed cameras, (2) the management of computational and storage resources using cloud computing, and (3) improved performance by reducing data movement, balancing loads among multiple cloud instances, and enhancing data-level parallelism. The project provides an application programming interface (API) that hides the underlying sophisticated infrastructure. This infrastructure will handle both real-time streaming data and archival data in a uniform way, so that the same analysis programs can be reused. This project has four major components: (1) a web-based user interface, (2) a database that stores the details about the network cameras, (3) a resource manager that allocates cloud instances, and (4) a computational engine that execute the programs written by users. The service-oriented architecture will allow new functions to be integrated more easily by the research community.

\begin{refsection}
    \nocite{purdue_cam2_gauen_visual_datasets,purdue_cam2_hst2018,purdue_cam2_kaseb_resource_management}
    \printbibliography[heading=none]
\end{refsection}

The HST 2018 conference paper received the Best Paper in Disaster Management award. Many other manuscripts are under consideration at major conferences, journals, and magazines.
